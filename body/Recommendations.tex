\chapter{Recommendations and Future Work}
\label{chap:recommendationsAndFutureWork}
  \section{Recommendations}
    \subsection{RCE Hardware Platform}
      As discussed in Section~\ref{subsec:rec-choiceOfRCEBoard}, the Intel Edison exhibited hardware-software limitations which negatively impacted the performance of some of the peripherals. A candidate solution would be to split the responsibility of the RCE between two devices. The Intel Edison performed well for scheduling, communication and video streaming and thus remains a suitable device for those responsibilities. A separate micro-controller module could be designed to accompany the Intel Edison, and would handle control of hardware and acquisition of data from hardware with the required performance and accuracy. The accompanying module could communicate with the Intel Edison via a serial interface such as I$^2$C and potentially alleviate the requirement for the Arduino extension module thus reducing the RCE boards spatial footprint.
      
    \subsection{Web Camera}
      The web camera chosen included many unnecessary features which were not required by the design. It is recommended that a smaller web camera module be used to decrease the required size of the head component and thus bring the camera and head sensor assembly to within proportions.
      
    \subsection{Proximity Sensors}
      The HC-SR04 ultrasonic sensors provided accurate proximity measurements, but were large in comparison to the majority of the body components. A smaller proximity sensor device would dominate less the overall aesthetic of the rover. If the Intel Edison is to be used for data acquisition, a proximity sensor with a digital interface would benefit the design.
      
    \subsection{Drive Servos}
      It is recommended to use servos designed for continuous rotation to ensure stability of the mounting of wheels. Servos with metal gears would also bring torque and robustness benefits to the driving of the wheels and the rover's traversal capabilities in general.
      
  \section{Future Work}
    \subsection{Rover Model as Mission and Operations Test Bed}
      During the development of the rover model, it became apparent that the model could be used in a more scientific manner. The rover could be used as a test bed for many of the systems and software algorithms for all aspects of the operation of this type of exploratory vehicle. This might include the testing of path planning algorithms or perhaps newly developed optical obstacle detection and avoidance systems with supporting automation.
      
    \subsection{Martian Environment Simulation}
    \label{subsec:fut-martianEnvironmentSimulation}
      As an extension to the project, research on methods of developing small scale and low-cost but accurate Martian environments could be done to support test-bed-like applications of such a rover model as well as improve the educational value and user engagement when operating the rover.
    
    \subsection{Steroscopic Video Feed}
      While the rover model in this project only required a monoscopic video stream, some of the cameras on \textit{Curiosity} are placed in pairs to provide stereoscopic imagery of the subject. This was designed for extraction of depth data as well as for operator immersion. A second camera could be added to the rover model in this project, along with modifications to the RCE board to allow for a second camera input, to provide a similar experience to the visualisation component of the RSVP used at JPL. 
    
    \subsection{Direct Control Capabilities}
      While it was deemed acceptable for the system to rely on a central server endpoint for the operation of the rover model in this project, improvements to the accessibility of operation could be achieved by designing a second mode of operation in which the server component of the software system is removed and the RSVP Clients communicate directly with the RCE system. The platform or device hosting the RSVP Client would connect to the RCE Board's wireless access point and the RCE would offer server endpoint functionality and video broadcast. Investigations into the performance limitations as a result would need to be made.
      
    \subsection{Improved RoSE Sequencing Editing}
      The RoSE control interface in this project was greatly simplified due to lack of time. Improvements could be made to the editor component of the RSVP Client UI such as the ability to organise sequences into folders or groups, save sequences or portions thereof for later use and the addition of many other commands for operation of all aspects of the rover.
    
    \subsection{Hardware Feedback Telemetry}
      Currently, the rover model does not have the ability to report the actual hardware states due to lack of sensors and other electronics required to obtain such data. The rover model could be extended to include sensors to provide more accurate state data to the RSVP Client and even to the RCE itself which could be used to improve the control of hardware and implement proper fault detection.