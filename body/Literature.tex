\chapter{Literature Review}
  \section{Space Exploration and NASA's Journey to Mars}
    The human race possesses a trait that proposedly sets us apart from life around us; the powerful will to explore what is unknown. It is the curiosity and the thrill to push past the boundaries of what is thought to be possible, perhaps felt stronger by some, that forms the basis of many scientific endeavours relating to facts of life and existence around and outside of the immediate environment in which we live.
    
    A prime example of such a drive to explore is in the research and exploration of outer space, which, from a technological perspective, transitioned from astronomer's dream to scientist's and engineer's reality during the Cold War. Although space exploration as we know it today is motivated by human curiosity, it was during this period of political tension that significant breakthroughs in spacecraft and rocket propulsion technology were brought about. This period is referred to as the ``Space Race'' and stemmed from research and development of nuclear weaponry during World War II \cite[p. 147]{cornwell2003hitler}. The race began with the attempted launches of artificially made satellites \cite[pp. 3-5]{schefter1999the} and within the 40 years following the success of the USSR's \textit{Sputnik I} in 1957, the first object to be put into orbit by man, space technology progressed from early manned flights beginning in 1961\footnote{First human in space, Soviet launched} through the \textit{Apollo 11} lunar flight to the flying by of the majority of the planets in our solar system.
    
    By 1981, the launch of Columbia \cite{williamharwood2009}, a space shuttle designed to be used for more than one flight, marked the beginning of reusable space technologies answering to the problem of cost and with the forethought of future increase in space flight frequency.
    
    % TODO: ISS
    % TODO: NASA and Mars
    
  \section{The Mars Science Laboratory}

  \section{Space Education and Outreach}
  
  \section{Web Application Technologies Within the Context of Embedded Systems}
  
  \section{Additive Prototyping and Manufacturing Techniques}