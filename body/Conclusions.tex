\chapter{Conclusions}
  In recognition of the potential of today's technology, specifically that of interconnectivity of devices and access of resources and services by many more than before, the project aimed to leverage advancements in prototyping, modern manufacture and the web in the design and development of a scaled down, working model of JPL's Mars Curiosity Rover, a major component of the Mars Science Laboratory Mission. The design setting was one whereby the model would be used for education in a modern-style, a highly interactive and accessibly means to allowing users to experience the nature of control of interplanetary rovers and insight into exploration of another planet.
  
  The design initiated with a review of \textit{Curiosity}, its mechanical design and instrumentation as well as supporting fields of engineering and other literature. A comprehensive review of the client requirements was performed to result in a list of technical specification upon which the rest of the design process was based. The project was, at this point, broken into multiple components which followed the design process in parallel. A conceptual design and development procedure was performed within each of the design components and this resulted in the choice of the all technologies and principles for each aspect of each component. The model was then designed in full detail which included a complete and dynamic 3D CAD model of the rover, electrical schematics and a plan of the software architecture. With the design completed, a BOM was drawn up and all components and manufacturing services sourced. The mechanical and electrical systems of the rover were assembled upon receipt of components and the 3D components that were outsourced. Simultaneously, the three software components (RCE, RSVP Server and RSVP Client) were developed and testing of these components was performed throughout the development phase.
  
  After a phase of integration of the developed software components and mechanical and electrical systems, which required multiple iterative developments to be made to all of the systems, the completed model was put through a series of tests which were in accordance with the technical specifications drawn up at the beginning of the project, of which the tests indicated the success of the design in these areas. The model and software system were then analysed against the list of specification which were individually verified. Discussion around specifications that were not fully satisfied led to candidate solutions and suggestions for rectification of any shortfall in the design.
  
  The project concluded with the successful design and development of the model of \textit{Curiosity} which was well suited to an educational environment and exhibited the effectiveness and benefits of modern design, electronics and web technologies in the context of education and outreach. Further developments were made that allowed the project to be open-sourced, making possible the replication of the project and providing insights into methodologies of the design of hardware and software for this type of project.