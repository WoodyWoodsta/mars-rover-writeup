\section{Post-development Verification of Specifications}
  Having performed the aforementioned tests on the rover model and the complete software system, a full post-development verification was performed in which the final product were analysed and verified against each of the requirements as outlined in Section~\ref{subsec:probDef-vehicleSpecifications}. The analysis aimed to determine if each requirement was satisfied and this was used as a platform for discussion of the entire design, development and the project in general.
  
  \begin{itemize}
    \item \textbf{Mechanical}
    \begin{RM}
      \item General Specifications:
      \begin{RM}
        \item \textit{Be as dimensionally accurate as possible to the \textit{Curiosity} rover on Mars.}\\
        \textbf{Partially Satisfied}\\
        All components of the vehicle were proportional to the 3D reference model \cite{nasa3Dprint} provided by NASA except in the mast and head assembly, whereby the servo and camera dimensions did not allow for smaller components, and in the beams of the suspension system.
        
        \textbf{Proposed improvements:}
        \begin{itemize}
          \item Sourcing of a smaller camera module would allow for a smaller camera head assembly together with better planning of the mounting of the camera inside of the head cavity.
          \item A smaller ultrasonic proximity sensor would alleviate the requirement for the extension of the head canopy for mounting purposes.
        \end{itemize}
        \item \textit{Exhibit a satisfactory level of realism and external aesthetic accuracy.}\\
        \textbf{Fully Satisfied} and further discussed in Section~\ref{subsec:rec-aesthetics}.
      \end{RM}
      \item Body:
      \begin{RM}
        \item \textit{Box shaped.}\\
        \textbf{Fully Satisfied}
        \item \textit{Allow mounting of the mast and differential on the top surface.}\\
        \textbf{Fully Satisfied}
        \item \textit{Allow mounting of the suspension system on either side.}\\
        \textbf{Fully Satisfied}
        \item \textit{Allow mounting of additional sensors.}\\
        \textbf{Fully Satisfied}
        \item \textit{Allow mounting of additional detail such as side panels and other mockup objects.}\\
        \textbf{Fully Satisfied}
        \item \textit{Allow mounting of electrical internals.}\\
        \textbf{Fully Satisfied}\\
        Mounting of the DC to DC Converter module and the pulse-to-analog converters was moved from the body to an additional acrylic piece fastened to the PWM extension module.
        \textit{Provide protection/coverage of electrical internals from the external environment.}\\
        \item \textbf{Fully Satisfied}
      \end{RM}
      \item Mast:
      \begin{RM}
        \item \textit{Provide mount point for the head module.}\\
        \textbf{Fully Satisfied}
        \item \textit{Facilitate full rotation about the $z$-axis (camera panning/yaw axis).}\\
        \textbf{Partially Satisfied}\\
        The range of actuation in the servo components chosen for the panning axis rotation of the head was limited to $180^\circ$.
        
        \textbf{Proposed Improvements:}
        \begin{itemize}
          \item Make use of a servo component capable of offering full rotation. This type of servo was not available at the time of design and development of the project.
        \end{itemize}
        \item \textit{Facilitate at least $120^\circ$ degrees rotation about the $y$-axis (camera pitching axis).}\\
        \textbf{Fully Satisfied}
        \item \textit{Be structurally secure providing robustness against lateral forces on the mounted head module.}\\
        \textbf{Partially Satisfied}\\
        Using the servo shaft for mounting reduced the structural stability of the assembly due to play in the plastic gears of the component. While the rigidity of the mast and head assembly was well within that which was required for operation, improved rigidity would have been possible with an alternative mounting configuration and/or use of metal gear servos. This is discussed further in Section~\ref{subsec:rec-servoSuitability}.
      \end{RM}
      \item Head:
      \begin{RM}
        \item \textit{Be mounted onto the mast module.}\\
        \textbf{Fully Satisfied}
        \item \textit{Provide mount point for a sensor.}\\
        \textbf{Fully Satisfied}
        \item \textit{Allow mounting of a camera module.}\\
        \textbf{Fully Satisfied}
      \end{RM}
      \item Suspension:
      \begin{RM}
        \item \textit{Ensure body stability.}\\
        \textbf{Fully Satisfied} as tested in \ref{test:obstacleTest}.
        \item \textit{Maintain stability despite uneven terrain. This includes terrain which might require asymmetrical articulation of the system (a rock underneath one side of the rover and flat terrain underneath the other).}\\
        \textbf{Fully Satisfied} as tested in \ref{test:obstacleTest}.
      \end{RM}
      \item Wheels and Hubs/Pivots:
      \begin{RM}
        \item \textit{Match the shape and proportion of the wheels on \textit{Curiosity}.}\\
        \textbf{Fully Satisfied}
        \item \textit{Provide traction required for an analogue Martian terrain.}\\
        \textbf{Fully Satisfied}
        While the traction capability of the designed wheels was not tested due to lack of a suitable test terrain, the wheels replicated the traction pattern on the tyres of \textit{Curiosity}.
        \item \textit{Have steering capabilities which would amount to arc pattern traversal as well as rotation of the rover around a central fixed point.}\\
        \textbf{Fully Satisfied} as tested in \ref{test:collisionTest}.
      \end{RM}
    \end{RM}
    
    \item \textbf{Electrical}
    \begin{RE}
      \item Actuation:
      \begin{RE}
        \item \textit{Provide continuous rotational actuation for driving. The actuation speed should be controllable and must be of high enough torque to satisfy traversal specifications.}\\
        \textbf{Fully Satisfied} as tested in \ref{test:stationaryTest} and \ref{test:obstacleTest}.
        \item \textit{Provide sufficient magnitude rotational actuation for turning of the wheels for steering. The rotational position should be controllable and must be of high enough torque to facilitate turning of the wheels in-place.}\\
        \textbf{Fully Satisfied} as tested in \ref{test:obstacleTest}.
        \item \textit{Provide required magnitude rotational actuation for panning and pitching of the head/mast module. Both axes must allow positional control.}\\
        \textbf{Fully Satisfied}
      \end{RE}
      \item Central Control:
      \begin{RE}
        \item \textit{Host onboard software system for control of hardware.}\\
        \textbf{Fully Satisfied}
        \item \textit{Facilitate wireless communications with the server.}\\
        \textbf{Fully Satisfied}
        \item \textit{Interface with actuation and sensory input hardware.}\\
        \textbf{Fully Satisfied} as partially tested in \ref{test:collisionTest}.
        \item \textit{Have performance capabilities sufficient for processing and streaming of video data from the head module.}\\
        \textbf{Fully Satisfied}
      \end{RE}
      
      Note that the choice of the Intel Edison was suitable for the designed rover, however, the severely limited control of the GPIO pins and other peripherals such as PWM could have been avoided if another device was chosen. This is further discussed in Section~\ref{subsec:rec-choiceOfRCEBoard}.
      
      \item Power:
      \begin{RE}
        \item \textit{Provide power for the central control hardware as well as actuation and sensor hardware components.}\\
        \textbf{Fully Satisfied}
        \item \textit{Power supply unit must fit inside or on the body module.}\\
        \textbf{Fully Satisfied}
        \item \textit{Have the ability to be turned switched off or on.}\\
        \textbf{Fully Satisfied}
        \item \textit{Allow convenient removal of source.}\\
        \textbf{Fully Satisfied}
        \item \textit{Allow easy access to charging ports.}\\
        \textbf{Fully Satisfied}
        \item \textit{Provide a means of indication of voltage for telemetry and low-battery warnings.}\\
        \textbf{Fully Satisfied}
      \end{RE}
      \item Sensors:
      \begin{RE}
        \item \textit{Provide immediate environment data required to implement elementary obstacle detection and avoidance.}\\
        \textbf{Partially Satisfied}\\
        Due to the limitations in the Intel Edison's ability to measure input electrical pulses, the pulse-to-analog conversion solution introduced a significant increase in the response time of the distance measurements. This meant that while satisfactory data was acquired, it was not immediate and thus affected the speed of obstacle detection.
        
        \textbf{Proposed Improvements:}
        \begin{itemize}
          \item Source the I$^2$C backpack designed to allow the Intel Edison to correctly interface with the HC-SR04 Sensors.
          \item Source digital proximity sensors or range-finders.
        \end{itemize} 
        \item \textit{Be mounted in locations similar to those on \textit{Curiosity}.}\\
        \textbf{Fully Satisfied}
        \item \textit{Be compatible with the central control module in terms of data interface.}\\
        \textbf{Not Satisfied} as discussed in the analysis of \ref{li:probDef-spec-sensors-immediateObstacleData}.
      \end{RE}
      \item Camera:
      \begin{RE}
        \item \textit{Facilitate a monoscopic video feed.}\\
        \textbf{Fully Satisfied}
        \item \textit{Be mountable inside the head module.}\\
        \textbf{Partially Satisfied}\\
        Post-manufacture modifications to the head canopy printed part had to be made to allow mounting of the camera. Due to the time-scale of the project, ordering of components and design of the components within the mechanical system had to occur simultaneously. The dimensions of the web camera module were unknown during the design of the mast assembly.
        \item \textit{Be compatible with the central control module in terms of data interface.}\\
        \textbf{Fully Satisfied}
      \end{RE}
  \end{RE}
\end{itemize}
  
\subsubsection{Software System Specifications}
  \begin{itemize}
    \item \textbf{Rover Embedded Software}
    \begin{RS}
      \item General Specifications:
      \begin{RS}
        \item \textit{Allow connection of a remote client for telemetry and control as well as another for video streaming.}\\
        \textbf{Fully Satisfied}
        \item \textit{Be robust against hardware errors and intermittent communication so as to maintain operation in these circumstances.}\\
        \textbf{Fully Satisfied}
      \end{RS}
      \item Control:
      \begin{RS}
        \item \textit{Provide a programmatic means to peripheral hardware access.}\\
        \textbf{Fully Satisfied}
        \item \textit{Translate control input commands into hardware output signals for control of peripheral hardware components.}\\
        \textbf{Fully Satisfied}
        \item \textit{Declare/define and execute programmatic sequences facilitating procedures such as system booting, communication initialisations, hardware initializations, self diagnosis and emergency modes.}\\
        \textbf{Fully Satisfied}
      \end{RS}
      \item Telemetry:
      \begin{RS}
        \item \textit{Emit system telemetry to the connected client consisting of system, process and hardware state as well as sequence execution notifications.}\\
        \textbf{Fully Satisfied}
      \end{RS}
      \item Video Stream:
      \begin{RS}
        \item \textit{Provide a stream of the video data to the connected client.}\\
        \textbf{Fully Satisfied}
        \item \textit{Provide video resolution on or above VGA (640$\times$480) resolution.}\\
        \textbf{Fully Satisfied}
      \end{RS}
    \end{RS}
    
    \item \textbf{Server}
    \begin{RS}[resume]
      \item General Requirements:
      \begin{RS}
        \item \textit{Manage communication with the rover system.}\\
        \textbf{Fully Satisfied}
        \item \textit{Manage communication with the connected web clients.}\\
        \textbf{Fully Satisfied}
        \item \textit{Serve web application to the connected web clients.}\\
        \textbf{Fully Satisfied}
        \item \textit{Manage roles of the connected web clients with respect to their level of access and ability to control the rover.}\\
        \textbf{Fully Satisfied}
      \end{RS}
      \item Video Broadcast:
      \begin{RS}
        \item \textit{Connect to and accept video data from the rover video stream.}\\
        \textbf{Fully Satisfied}
        \item \textit{Broadcast video stream to a scalable number of connected user clients.}\\
        \textbf{Fully Satisfied} as tested in \ref{test:serverLoadTest}.
        \item \textit{Be robust against communication intermittency in terms of connection with the rover system.}\\
        \textbf{Fully Satisfied}
      \end{RS}
      \item Data Relay:
      \begin{RS}
        \item \textit{Relay control input commands from a controlling web client to the rover.}\\
        \textbf{Fully Satisfied}
        \item \textit{Provide a means of simulating long-distance communication.}\\
        \textbf{Partially Satisfied}\\
        The means by which long distance communication was simulated was rudimentary in that it consisted only of a time delay below 60 seconds. This was not an accurate depiction of the communication dynamic between Earth and \textit{Curiosity}, however, such a simulation would have detracted from the experience of the user.
        \item \textit{Relay telemetry data from the rover system to the connected web clients.}\\
        \textbf{Fully Satisfied}
        \item \textit{Relay state information of the rover system and server system to the connected web clients.}\\
        \textbf{Fully Satisfied}
      \end{RS}
    \end{RS}
      
    \item \textbf{Client}
    \begin{RS}[resume]
      \item Control:
      \begin{RS}
        \item \textit{Provide two means of control of the rover, if access is granted.}\\
        \textbf{Fully Satisfied}
      \end{RS}
      \item Telemetry:
      \begin{RS}
        \item \textit{Accept and display telemetry received from the rover via the server and from the server itself.}\\
        \textbf{Fully Satisfied}
      \end{RS}
      \item Video Feed:
      \begin{RS}
        \item \textit{Accept and display video feed received from the broadcast.}\\
        \textbf{Fully Satisfied}
      \end{RS}
    \end{RS}
  \end{itemize}
  
\section{Discussion}
  \subsection{Servos as the Mechanism for Actuation}
    \label{subsec:rec-servoSuitability}
    The Hextronik HXT900 servos were chosen for all types of actuation, including driving of the wheels which required the modification of the internal control electronics and gearing to allow for continuous rotation. During the testing of the model, it was found that the modifications made to the shaft and output gear assembly introduced a significant amount of instability to the splined shaft of the servo. The result of this decrease in stability was observed in the tests performed whereby the drive wheels did not maintain horizontal alignment. It was also found that, in general, the shaft servo-horn mounts exhibited a noticeable amount flexure under small forces. This was unanticipated, however, it did not prevent the successful mounting of the head and neck assembly, wheel pivot and strut assemblies and that of the wheels.
    
    The HXT900 servos performed well for the steering component of actuation with satisfactory response time and torque.
  
  \subsection{Choice of RCE Board}
    \label{subsec:rec-choiceOfRCEBoard}
    The Intel Edison together with the Arduino Breakout extension board was the hardware platform of choice for the RCE. During the conceptual design and development stage of the project, it was anticipated to provide the required performance and degree of connectivity for the project, as well as offered the required hardware interfaces. However, during design and development of the RCE, difficulties were encountered with the hardware as well as in the platform form-factor and technology stack in general. The Intel Edison Arduino Breakout extension board contained multiple design flaws which were poorly documented and addressed.
    
    Difficulties were also encountered in the general platform technology stack, being the use of Node.js and Linux in the context of embedded hardware and control. The Intel Edison's operating system stack prevented the low-level hardware access that would have been convenient in terms of the control subcomponent of the RCE. The use of \mintinline{js}{sysfs} as a means of software-hardware interface subjected the control to the non-determinism of the Linux operating system such as with the inability to accurately time pulse inputs from the HC-SR04 sensors. \mintinline{js}{sysfs} distanced the execution of user code from the control of hardware thus making the Intel Edison better suited to applications with more emphasis on embedded connectivity as opposed to the control of hardware and acquisition of data. However, this was not to be incorrectly contrasted with the benefits of being able to use JavaScript and Node.js together with the plethora of open-source software in the embedded context.
  
  \subsection{Aesthetic Accuracy and Model Realism}
  \label{subsec:rec-aesthetics}
    The use of 3D printing for the majority of the components allowed for a high degree of realism as well as provided better than expected strength and suitability from a mechanical perspective. Various aspects of the model could have been improved in terms of aesthetic accuracy (\ref{li:probDef-spec-mechanical-realism}). The use of M3 sized fasteners for the majority of the mounting took away from the realism of the model and was identified as an over-design as far as structural stability was concerned. The use of M2 fasteners (or smaller) would have improved the aesthetics of the rover and still been a suitable means to fastening of components. The use of nut and bolt fasteners for the aluminium tubing as part of the suspension system also detracted from the realism of this system and could have been replaced with set screws for a less prominent appearance.
    
    However, the objective of the model design was for it to be accurate enough to be easily recognised as a replication of \textit{Curiosity} and this was successfully achieved.
  