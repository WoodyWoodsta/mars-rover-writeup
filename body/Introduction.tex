\chapter{Introduction}
\section{Background to the study}
  Space exploration, specifically of planets in our solar system, have become an active and exciting field of research, gaining support from public and private entities. There are many reasons behind rekindling of interest in space, including acquisition of resources, communication infrastructure, transport and even entertainment and leisure. However, one of the more high-reaching goals driving the study of space is the potential for humanity to inhabit more than one planet. An important requirement for such an existence to become a reality is for there to be knowledge of a planet that would be capable of supporting life. Observation into deep space is an ongoing effort in this regard, however, today's technology has limited contact observation of other planets to those within our solar system. This makes Mars a relatively superior candidate due to its proximity and its similarity in surface properties to those of Earth. As a result, many scientist have placed focus on Mars in terms of their research and once such endeavour is the use of rover vehicles to explore the surface of the planet. Over the last 30 years, rovers have been sent to Mars in an attempt to make hand-on observations of surface features, investigations into the atmospheric composition, climatic characteristics and signs of life existing already. To this day, the rovers have proved to be a successful means of observation and are a continued avenue in the area of planetary exploration.
  
  Once such rover, developed by JPL and administrated by NASA, is the Mars \textit{Curiosity} Rover as part of the Mars Science Laboratory Mission; a scientific effort with strong emphasis on searching for signs of life as indicators of suitability for human inhabitance. At the time of writing, \textit{Curiosity} is still in operation on Mars and has been for over four Earth-years. It is the most advanced rover to explore a planet, comprising of high-fidelity imaging systems, hardware for scientific experiments and a host of sensory devices.
  
\section{Objectives of this study}
  This project recognises the importance of rovers, \textit{Curiosity} in particular, in exploration of this type and stems from the identification of the lack of awareness on the part of the general public. Such a progression in science and further in the form of existence of humankind calls for the promotion of this science; one which could greatly benefit from the support of the public. The project also leveraged the potential of modern day advancements and growth in accessibility of informational resources and platforms and aimed to make use of these developments to bring awareness to the operation and control of \textit{Curiosity} and similar rovers in the form of a fully functional and remotely controlled model.
  
  Emphasis is placed on accessibility and interactivity during the design and development of this model to ensure suitability of the model in the context of education and outreach. The project also aims to serve as a proof-of-concept of modern connected devices and technologies in this field as well as promote the use of open source developments and collaboration.

\section{Scope and Limitations}
  The scope of the project is bound to the design and development of the replication of mobility, imagery and spatial awareness systems of \textit{Curiosity} at a predefined scale and at an appropriate level of detail. The project also includes the development of the software systems required to remotely control the model in a way which portrays the control and operation of the real \textit{Curiosity} at JPL.
  
  The project is limited to the communication infrastructure for which the model is designed and the computational platforms for which the software systems were intended to support. The project is also limited for the purpose of maintaining ease of replication of the final product by third parties for their own benefits.
  
  The design was ultimately limited by the available resources and budget, local availability of hardware and electronic components and methods of manufacture.
  
\section{Plan of development}
  The design and development of the rover simulator was initiated with a comprehensive review of literature and other material on the \textit{Curiosity} rover and planetary exploration in general. The allowed for the compilation of a set of features deemed worthy of inclusion in the final product. During the collation of candidate features, client requirements were factored into the process from which a list of detailed technical specifications were formed aiming to cover all areas of the project. The project was componentised by nature of design and engineering discipline, the classification of which reflected in that of the specifications. Each component then followed a classical process of engineering design in which conceptual solution candidates were developed and explored resulting in a comparative analysis of each. The comparative analysis aided the final choice of technologies and principles and these were used to design, in detail, all aspects of the final product. Once the detailed designs were complete, they were used to fully develop the final product which was tested and verified against the list of specifications. Conclusions were then drawn up based on the analysis and recommendations were made for future work.
  
\section{Report Outline}
  This report covers the processes as described in the plan of development, following the order in which they were introduced. The report structure is outlined in Table~\ref{tab:intrp-reportStructure}.
  
  \begin{longtable}{@{}L{0.15\textwidth}L{0.2\textwidth}L{0.6\textwidth}@{}}
  \toprule
  \textbf{Chapter(s) and/or Section(s)}                                                       & \textbf{Project Stage}                             & \textbf{Description}                                                                                                                                                                                                                                                                                                                                                                                            \\ \midrule
  Chapter~\ref{chap:lit-review}                                                    & Review of literature                      & The entire chapter covers the review of literature on the history of the research into and exploration of space and Mars's place in this history. Research on the Mars Curiosity Rover is covered and the chapter ends off with a brief look into web technologies in the context of education and outreach as well as existing rover models.                                                          \\ \midrule
  Section~\ref{sec:probDef-devObjectives}                                          & Problem Definition                        & The problem definition sections include introduction of the problem as well as the client requirements. The functional breakdown and analysis is covered after which the technical specifications are listed.                                                                                                                                                                                          \\ \midrule
  Section~\ref{sec:conceptualDesign}                                               & Conceptual Development                    & Each of the conceptual development processes of each of the components of design are covered after which the final design choice and all technologies within are outlined and discussed.                                                                                                                                                                                                               \\ \midrule
  Sections~\ref{sec:detailedDesign} to \ref{sec:softwareDesign}                  & Detailed Design                           & Detailed designs of each of the three project systems, mechanical, electrical and software, are covered chronologically in that order. After the design of each group of individual components is discussed, sub-assemblies or completed modules are outlined where applicable.                                                                                                                        \\ \midrule
  Sections~\ref{sec:vehicleBuildAndManufacture} to \ref{sec:softwareDevelopment} & Development and Manufacture               & The processes followed in developing and manufacturing the final product using the detailed designs is covered in this set of sections. The mechanical and electrical manufacture processes are dealt with first which included manufacturing plans and a bill of materials. The software development follows where significant areas of the large software system are covered with snippets included. \\ \midrule
  Chapter~\ref{chap:roverPostDev}                                                  & Post-development Testing and Verification & In this chapter, post-development procedures are covered including the testing of the model in typical scenarios and a full verification of the final product against the technical specifications is included. The verification of specifications was used as a platform from which significant areas of the project were discussed.                                                                  \\ \midrule
  Chapters~\ref{chap:conclusions} to \ref{chap:recommendationsAndFutureWork}     & Conclusions and Reccomendations           & In the final chapters, conclusions of the project that were drawn are covered and the recommendations formulated as part of the discussions are highlighted. Potential avenues for future work are also included.                                                                                                                                                                                      \\ \bottomrule
  \caption{Description of the structure of the report as per the stages of design and development of the project.}
  \label{tab:intrp-reportStructure}
  \end{longtable}