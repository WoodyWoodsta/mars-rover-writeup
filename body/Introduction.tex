\chapter{Introduction}

\section{Background to the study}
A very brief background to your area of research. Start off with a general introduction to the area and
then narrow it down to your focus area. Used to set the scene \cite{smt2011}. The section should highlight challenges in the study area to put your work in context \cite{kamen}.
\section{Objectives of this study}
\subsection{Problems to be investigated}
Description of the main problem(s) to be solved and/or hypothesis of your work. Questions to be answered in order to confirm the hypothesis or solve the problems are also articulated here.
\subsection{Purpose of the study}
Give the significance of investigating these problems. It must be obvious why you are doing this study
and why it is relevant. Contributions of your work should also be given here.

\section{Scope and Limitations}
Scope indicates to the reader what has been and not been included in the study. Limitations tell the
reader what factors influenced the study such as sample size, time etc. It is not a section for excuses as
to why your project may or may not have worked.

\section{Plan of development}
This section summarizes the methods, tools, techniques and the order of doing things followed in order to accomplish your work. It also includes such planning tools as project Gantt chart, Critical path analysis and mind mapping.

\section{Report Outline}
Here you tell the reader how your report has been organised and what is included in each
chapter. You should give a synopsis for each of your chapters here.

{\bf I recommend that you write this section last. You can then tailor it to your report.}
